
\input{../../style/preamble}
\input{../../latex-math/basic-math}
\input{../../latex-math/basic-ml}
\input{../../latex-math/ml-nn}

\newcommand{\titlefigure}{figure/backprop_gg_new.png}
\newcommand{\learninggoals}{
  \item Chain Rule of Calculus
  \item Computational Graphs
}

\title{Deep Learning}
\date{}

\begin{document}

\lecturechapter{Chain Rule and Computational Graphs}
\lecture{I2DL}

%%%%%%%%%%%%%%%%%%%%%%%%%%%%%%%%%%%%%%%%%%%%%%%%%%%%%%%%%%%%%%%%%%

%\begin{frame}
%\frametitle{Lecture outline}
%\tableofcontents
%\end{frame}
%%%%%%%%%%%%%%%%%%%%%%%%%%%%%%%%%%%%%%%%%%%%%%%%%%%%%%%%%%%%%%%%%%
% \begin{frame}{Local Minima/Convexity}
% \end{frame}

% \section{Chain Rule and Computational Graphs}
%%%%%%%%%%%%%%%%%%%%%%%%%%%%%%%%%%%%%%%%%%%%%%%%%%%%%%%%%%%%%%%%%%
\begin{vbframe}{Chain rule of calculus}
  % \begin{itemize}
  %   \item The chain rule can be used to compute derivatives of the composition of two or more functions.
  %   \item Let $x \in \R^m$, $y \in \R^n$, \\
  %         $g: \R^m \to \R^n$ and $f: \R^n \to \R$. \\
  %   \item If $y = g(x)$ and $z = f(y)$, the chain rule yields $$\frac{\partial z}{\partial x_i} = \sum_j \frac{\partial z}{\partial y_j} \frac{\partial y_j}{\partial x_i}$$
  %         or in vector notation $$\nabla_x z = \Big(\frac{\partial y}{\partial x}\Big)^\top \nabla_y z,$$
  %         where $\frac{\partial y}{\partial x}$ is the $n \times m$ jacobian matrix of $g$.
  % \end{itemize}
  \begin{itemize}
    \item The chain rule can be used to compute derivatives of the composition of two or more functions.
    \item Let $\xb \in \R^m$, $\mathbf{y} \in \R^n$, \\
          $g: \R^m \to \R^n$ and $f: \R^n \to \R$. \\
    \item If $\mathbf{y} = g(\xb)$ and $z = f(\mathbf{y})$, the chain rule yields: $$\frac{\partial z}{\partial x_i} = \sum_j \frac{\partial z}{\partial y_j} \cdot \frac{\partial y_j}{\partial x_i}$$
          or, in vector notation: $$\nabla_{\xb} z = \Big(\frac{\partial \mathbf{y}}{\partial \xb}\Big)^\top \nabla_{\mathbf{y}} z,$$
          where $\frac{\partial \mathbf{y}}{\partial \xb}$ is the ($n \times m$) Jacobian matrix of $g$.
  \end{itemize}
\end{vbframe}  

\begin{vbframe}{Computational graphs}
  \begin{minipage}{0.45\textwidth}
    \begin{itemize}
      \item Computational graphs are a very helpful language to understand and visualize the chain rule.
      \item Each node describes a variable.
      \item Operations are functions applied to one or more variables.
    \end{itemize}
  \end{minipage}\hfill
  \begin{minipage}{0.5\textwidth}
    \begin{figure}
      \centering
        \includegraphics[width=6cm]{figure/compgraph1.png}
        \tiny{\\source : Goodfellow et al. (2016)}
        \caption{The computational graph for the expression $H = \sigma(XW + B)$ with activation function $\sigma(\cdot)$.}
    \end{figure}
  \end{minipage}  
\end{vbframe}

\begin{vbframe}{Chain rule of calculus: Example 1}
  \begin{minipage}{0.5\textwidth}
    \begin{itemize}
      \item Suppose we have the following computational graph.
      \item To compute the derivative of $\frac{\partial z}{\partial w}$ %$\frac{\partial z}{\partial w}$ 
      we need to recursively apply the chain rule. That is:
      \begin{eqnarray*}
        \frac{\partial z}{\partial w} &=& \frac{\partial z}{\partial y} \cdot \frac{\partial y}{\partial x} \cdot \frac{\partial x}{\partial w} \\
                                  &=& f'_3(y) \cdot f'_2(x) \cdot f'_1(w) \\
                                  &=& f'_3(f_2(f_1(w))) \cdot f'_2(f_1(w)) \cdot f'_1(w)
      \end{eqnarray*}
    \end{itemize}
  \end{minipage}\hfill
  \begin{minipage}{0.32\textwidth}
    \begin{figure}
      \centering
        \includegraphics[width=1cm]{figure/compgraph2.png}
        \begin{footnotesize}
        \tiny{\\source : Goodfellow et al. (2016)}
        \caption{A computational graph, such that $x = f_1(w),$ $y = f_2(x)$ and $z = f_3(y)$.}
        \end{footnotesize}
    \end{figure}
  \end{minipage}
% \framebreak
%   \begin{figure}
%     \centering
%       \includegraphics[width=4.5cm]{figure/compgraph3.png}
%       \caption{Applying the chain rule to the example yields us a computational graph with a symbolic description of the
% derivatives.}
%   \end{figure}  
\end{vbframe}

\begin{frame}{Chain rule of calculus: Example 2}

   \begin{figure}
    \centering
      \scalebox{0.27}{\includegraphics{figure/chain_tree.png}}
  \end{figure}
  
To compute $\nabla_\xb z$, we apply the chain rule % to the computational graph above,
  \begin{itemize}
    \item $\frac {\partial z}{\partial x_1} = \sum_j \frac{\partial z}{\partial y_j} \frac{\partial y_j}{\partial x_1} = \frac {\partial z}{\partial y_1} \frac {\partial y_1}{\partial x_1} + \frac {\partial z}{\partial y_2} \frac {\partial y_2}{\partial x_1}$
    \item $\frac {\partial z}{\partial x_2} = \sum_j \frac{\partial z}{\partial y_j} \frac{\partial y_j}{\partial x_2} = \frac {\partial z}{\partial y_1} \frac {\partial y_1}{\partial x_2} + \frac {\partial z}{\partial y_2} \frac {\partial y_2}{\partial x_2}$
  \end{itemize}
  \vspace{2mm}
    Therefore, the gradient of $z$ w.r.t $\xb$ is
    \begin{itemize}
      \item  $\nabla_\xb z = \begin{bmatrix}
               \frac {\partial z}{\partial x_1} \\
               \frac {\partial z}{\partial x_2} \\
             \end{bmatrix} = \underbrace{\begin{bmatrix} \frac{\partial y_1}{\partial x_1}&\frac {\partial y_2}{\partial x_1}\\
                                             \frac {\partial y_1}{\partial x_2}&\frac {\partial y_2}{\partial x_2}\\
             \end{bmatrix}}_{\textcolor{red}{(\frac{\partial \mathbf{y}}{\partial \xb})^\top}} \underbrace{\begin{bmatrix} \frac {\partial z}{\partial y_1} \\
            \frac {\partial z}{\partial y_2} \\ \end{bmatrix}}_{\textcolor{red}{\nabla_{\mathbf{y}} z}} = \Big(\frac{\partial \mathbf{y}}{\partial \xb}\Big)^\top \nabla_{\mathbf{y}} z $
  \end{itemize}
\end{frame}

\begin{frame} {Computational Graph: Neural Net}
  \begin{figure}
      \centering
        \scalebox{0.75}{\includegraphics{figure/neo_comp.png}}
        \caption{A neural network can be seen as a computational graph. $\phi$ is the weighted sum and $\sigma$ and $\tau$ are the activations. \\
        Note: In contrast to the top figure, the arrows in the computational graph below merely indicate \textbf{dependence}, not weights.}
    \end{figure}
\end{frame}
% \begin{frame}{Chain rule of calculus}
% \begin{figure}
%     \centering
%       \scalebox{0.4}{\includegraphics{figure/chain_tree.png}}
%   \end{figure}
%   
%     $\nabla_x z = \begin{bmatrix}
%            \frac {\partial z}{\partial x_1} \\
%            \frac {\partial z}{\partial x_2} \\
%          \end{bmatrix} = \begin{bmatrix}
%            \frac {\partial z}{\partial x_1} \\
%            \frac {\partial z}{\partial x_2} \\
%          \end{bmatrix}$
% \end{frame}


%%%%%%%%%%%%%%%%%%%%%%%%%%%%%%%%%%%%%%%%%%%%%%%%%%%%%%%%%%%%%%%%%%
%%%%%%%%%%%%%%%%%%%%%%%%%%%%%%%%%%%%%%%%%%%%%%%%%%%%%%%%%%%%%%%%%%
% \begin{frame} {Simple Example?}
%   \begin{figure}
%     \centering
%       \scalebox{1}{\includegraphics{figure/simex.png}}
%   \end{figure}
% \end{frame}
%%%%%%%%%%%%%%%%%%%%%%%%%%%%%%%%%%%%%%%%%%%%%%%%%%%%%%%%%%%%%%%%%%

%%%%%%%%%%%%%%%%%%%%%%%%%%%%%%%%%%%%%%%%%%%%%%%%%%%%%%%%%%%%%%%%%%
%%%%%%%%%%%%%%%%%%%%%%%%%%%%%%%%%%%%%%%%%%%%%%%%%%%%%%%%%%%%%%%%%%
%%%%%%%%%%%%%%%%%%          REFERENCES          %%%%%%%%%%%%%%%%%%
%%%%%%%%%%%%%%%%%%%%%%%%%%%%%%%%%%%%%%%%%%%%%%%%%%%%%%%%%%%%%%%%%%

%\section{References}

\begin{vbframe}
\frametitle{References}
\footnotesize{
\begin{thebibliography}{99}

%%%%%%%%%%%%%%%%%%%%%%%%%%%%%%%%%%
\bibitem[Ian Goodfellow et al., 2016]{1} Ian Goodfellow, Yoshua Bengio and Aaron Courville (2016)
\newblock Deep Learning
\newblock \emph{\url{http://www.deeplearningbook.org/}}
%%%%%%%%%%%%%%%%%%%%%%%%%%%%%%%%%%

\end{thebibliography}
}
\end{vbframe}
%%%%%%%%%%%%%%%%%%%%%%%%%%%%%%%%%%%%%%%%%%%%%%%%%%%%%%%%%%%%%%%%%%
%%%%%%%%%%%%%%%%%%%%%%%%%%%%%%%%%%%%%%%%%%%%%%%%%%%%%%%%%%%%%%%%%%

\endlecture
\end{document}