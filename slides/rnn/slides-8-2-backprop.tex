

\input{../../style/preamble}
\input{../../latex-math/basic-math}
\input{../../latex-math/basic-ml}
\input{../../latex-math/ml-nn}

\newcommand{\titlefigure}{figure/backpropagation_rnn.png}
%modify picture
\newcommand{\learninggoals}{
  \item How does Backpropagation work for RNNs?
  \item Exploding and Vanishing Gradients
}

\title{Deep Learning}
\date{}



\begin{document}

\lecturechapter{Recurrent Neural Networks - Backpropogation}
\lecture{I2DL}

%%%%%%%%%%%%%%%%%%%%%%%%%%%%%%%%%%%%%%%%%%%%%%%%%%%%%%%%%%%%%%%%%%
%%%%%%%%%%%%%%%%%%%%%%%%%%%%%%%%%%%%%%%%%%%%%%%%%%%%%%%%%%%%%%%%%%


%%%%%%%%%%%%%%%%%%%%%%%%%%%%%%%%%%%%%%%%%%%%%%%%%%%%%%%%%%%%%%%%%%
%%%%%%%%%%%%%%%%%%%%%%%%%%%%%%%%%%%%%%%%%%%%%%%%%%%%%%%%%%%%%%%%%%


\frame{

\frametitle{Backpropagation through time}

  \center
  \includegraphics[width=6cm]{figure/backpropagation_rnn.png}

  %\begin{itemize}

    %\item For training the RNN we need to compute $\frac{d L}{d u_{i,j}}$, $\frac{d L}{d v_{i,j}}$, and $\frac{d L}{d w_{i,j}}$.
    %\item To do so, during backpropagation at time step $t$ %for an arbitrary RNN, 
    %we may need to compute
    $$\frac{d L}{d \bm{z}^{[1]}} = \frac{d L}{d \bm{z}^{[t]}} \frac{d \bm{z}^{[t]}}{d \bm{z}^{[t-1]}} \dots \frac{d \bm{z}^{[2]}}{d \bm{z}^{[1]}}$$

  %\end{itemize}

}


%%%%%%%%%%%%%%%%%%%%%%%%%%%%%%%%%%%%%%%%%%%%%%%%%%%%%%%%%%%%%%%%%%
%%%%%%%%%%%%%%%%%%%%%%%%%%%%%%%%%%%%%%%%%%%%%%%%%%%%%%%%%%%%%%%%%%
% \begin{vbframe}{Nlp - Example}
%       Example: 
%   \begin{itemize}
%     \item Suppose we only had a vocabulary of four possible letters: \enquote{h}, \enquote{e}, \enquote{l} and \enquote{o}
%     \item We want to train an RNN on the training sequence \enquote{hello}.
%     \item This training sequence is in fact a source of 4 separate training examples:
%       \begin{itemize}
%         \item The probability of \enquote{e} should be likely given the context of \enquote{h}
%         \item \enquote{l} should be likely in the context of \enquote{he}
%         \item \enquote{l} should also be likely given the context of \enquote{hel}
%         \item and \enquote{o} should be likely given the context of \enquote{hell}
%       \end{itemize}
%   \end{itemize}
% \end{vbframe}
%%%%%%%%%%%%%%%%%%%%%%%%%%%%%%%%%%%%%%%%%%%%%%%%%%%%%%%%%%%%%%%%%%
%%%%%%%%%%%%%%%%%%%%%%%%%%%%%%%%%%%%%%%%%%%%%%%%%%%%%%%%%%%%%%%%%%
% \frame{
% 
% \frametitle{Nlp - Example}
% 
% The RNN has a 4-dimensional input and output. The exemplary hidden layer consists of 3 neurons. This diagram shows the activations in the forward pass when the RNN is fed the characters \enquote{hell} as input. The output contains confidences the RNN assigns for the next character.
%   \begin{itemize}
%     \item[]
%   \end{itemize}
%   \begin{minipage}{0.55\textwidth}
%     \begin{figure}
%         \only<1>{\includegraphics[width=5.5cm]{plots/nlp1.png}}%
%         \only<2>{\includegraphics[width=5.5cm]{plots/nlp2.png}}%
%     \end{figure}
%   \end{minipage}%\hfill
%   \begin{minipage}{0.45\textwidth}
%   %\vspace{-0.3cm}
%   
%     \begin{itemize}
%       \only<1>{\item[] \textcolor{white}{Our goal is to increase the confidence for the correct letters (green digits) and decrease the confidence of all others (we could also use a softmax activation to squash the digits to probabilities $\in [0,1]$).}} 
%       \only<2>{\item[] Our goal is to increase the confidence for the correct letters (green digits) and decrease the confidence of all others (we could also use a softmax activation to squash the digits to probabilities $\in [0,1]$).} 
%     \end{itemize}
%   \end{minipage}
%   
% }
%%%%%%%%%%%%%%%%%%%%%%%%%%%%%%%%%%%%%%%%%%%%%%%%%%%%%%%%%%%%%%%%%%
%%%%%%%%%%%%%%%%%%%%%%%%%%%%%%%%%%%%%%%%%%%%%%%%%%%%%%%%%%%%%%%%%%
% \begin{vbframe}{Word Embeddings (Word2vec)}
%   \begin{minipage}{0.4\textwidth}
%     \begin{itemize}
%       \item Data Sparsity: 
%       \item[]
%       $$\text{man} \to \begin{bmatrix}
%                                   0\\
%                                   \vdots\\
%                                   0\\
%                                   1\\
%                                   0\\
%                                   \vdots\\
%                                   0
%                         \end{bmatrix} \to
%                         \begin{bmatrix}
%                                   0.35\\
%                                   -0.83\\
%                                   \vdots\\
%                                   0.11\\
%                                   3.2
%                         \end{bmatrix}
%       $$
%   
%     \end{itemize}
%   \end{minipage}
%   \begin{minipage}{0.55\textwidth}
%     \begin{itemize}
%       \item[]
%     \end{itemize}
%     \begin{figure}
%       \includegraphics[width=4.5cm]{plots/word2vec.png}%
%       \caption*{https://www.tensorflow.org/tutorials/word2vec/}
%     \end{figure}    
%   \end{minipage}
% \end{vbframe}
%%%%%%%%%%%%%%%%%%%%%%%%%%%%%%%%%%%%%%%%%%%%%%%%%%%%%%%%%%%%%%%%%%
%%%%%%%%%%%%%%%%%%%%%%%%%%%%%%%%%%%%%%%%%%%%%%%%%%%%%%%%%%%%%%%%%%
\begin{vbframe}{Long-Term Dependencies}
  
  \begin{itemize}
    \item Here, $\bm{z}^{[t]} = \sigma(\bm{V}^\top \bm{z}^{[t-1]} + \bm{W}^\top \xv^{[t]} + \bm{b})$
    \item It follows that:
    $$ \frac{d \bm{z}^{[t]}}{d\bm{z}^{[t-1]}} = \text{diag}(\sigma' (\bm{V}^\top \bm{z}^{[t-1]} + \bm{W}^\top \xv^{[t]} + \bm{b})) \bm{V}^\top = \bm{D}^{[t-1]} \bm{V}^\top $$
    $$ \frac{d \bm{z}^{[t-1]}}{d\bm{z}^{[t-2]}} = \text{diag}(\sigma' (\bm{V}^\top \bm{z}^{[t-2]} + \bm{W}^\top \xv^{[t-1]} + \bm{b})) \bm{V}^\top= \bm{D}^{[t-2]} \bm{V}^\top $$
    $$ \vdots $$
    $$ \frac{d \bm{z}^{[2]}}{d\bm{z}^{[1]}} = \text{diag}(\sigma' (\bm{V}^\top \bm{z}^{[1]} + \bm{W}^\top \xv^{[2]} + \bm{b})) \bm{V}^\top = \bm{D}^{[1]} \bm{V}^\top $$
    
    $$ \frac{d L}{d \bm{z}¸^{[1]}} = \frac{d L}{d \bm{z}^{[t]}} \frac{d \bm{z}^{[t]}}{d \bm{z}^{[t-1]}} \dots \frac{d \bm{z}^{[2]}}{d \bm{z}^{[1]}} = \bm{D}^{[t-1]} \bm{D}^{[t-2]}   \text{ . . . } \bm{D}^{[1]} (\bm{V}^\top)^{t-1}$$
%    $$ \frac{d \bm{z}^{[1]}}{d\bm{x}^{[1]}} = \text{diag}(\sigma' (\bm{V}^\top \bm{z}^{[0]} + \bm{W}^\top \xv^{[1]} + \bm{b})) \bm{W}^\top$$
    \item In general, for an arbitrary time-step $i<t$ in the past, $\frac{d\bm{z}^{[t]}}{d\bm{z}^{[i]}}$ will contain the term $(\bm{V}^\top)^{t-i}$ (this follows from the chain rule).
    \item Based on the largest eigenvalue of $\bm{V}^\top$, the presence of the term $(\bm{V}^\top)^{t-i}$ can either result in vanishing or exploding gradients.
    \item This problem is quite severe for RNNs (as compared to feedforward networks) because the \textbf{same} matrix $\bm{V}^\top$ is multiplied several times. \href{https://tinyurl.com/vangrad}{\beamergotobutton{Click here}}
    \item As the gap between $t$ and $i$ increases, the instability worsens.
    \item It is thus quite challenging for RNNs to learn long-term dependencies. The gradients either \textbf{vanish} (most of the time) or \textbf{explode} (rarely, but with much damage to the optimization).
    \item That happens simply because we propagate errors over very many stages backwards.
  \end{itemize}
  
  \framebreak
  \begin{figure}
      \centering
      \scalebox{0.8}{\includegraphics{figure/rnn_exploding.png}}
      \caption{\footnotesize {Exploding gradients}}
  \end{figure}

  \begin{itemize}
    \item Recall, that we can counteract exploding gradients by implementing gradient clipping.
    \item To avoid exploding gradients, we simply clip the norm of the gradient at some threshold $h$ (see chapter 4): $$\text{if  } ||\nabla W|| > \text h: \nabla W \leftarrow \frac{h}{||\nabla W||} \nabla W $$
  \end{itemize}

\framebreak

  \begin{figure}
      \centering
      \scalebox{0.8}{\includegraphics{figure/rnn_vanishing.png}}
      \caption{\footnotesize {Vanishing gradients}}
  \end{figure}


  \begin{itemize}
    \item Even for a stable RNN (gradients not exploding), there will be exponentially smaller weights for long-term interactions compared to short-term ones and a more sophisticated solution is needed for this vanishing gradient problem (discussed in the next chapters).
    \item The {vanishing gradient problem} heavily depends on the choice of the activation functions.
    \begin{itemize} 
      \item Sigmoid maps a real number into a \enquote{small} range (i.e. $[0, 1]$) and thus even huge changes in the input will only produce a small change in the output. Hence, the gradient will be small.
      \item This becomes even worse when we stack multiple layers.
      \item We can avoid this problem by using activation functions which do not \enquote{squash} the input.
      \item The most popular choice is ReLU with gradients being either $0$ or $1$, i.e., they never saturate and thus don't vanish.
      \item The downside of this is that we can obtain a \enquote{dead} ReLU.
    \end{itemize}

  \end{itemize}


\end{vbframe}
%%%%%%%%%%%%%%%%%%%%%%%%%%%%%%%%%%%%%%%%%%%%%%%%%%%%%%%%%%%%%%%%%%
%%%%%%%%%%%%%%%%%%          REFERENCES          %%%%%%%%%%%%%%%%%%
%%%%%%%%%%%%%%%%%%%%%%%%%%%%%%%%%%%%%%%%%%%%%%%%%%%%%%%%%%%%%%%%%%
\begin{vbframe}
\frametitle{References}
\footnotesize{
\begin{thebibliography}{99}
%%%%%%%%%%%%%%%%%%%%%%%%%%%%%%%%%%
\bibitem[Ian Goodfellow et al., 2016]{1} Ian Goodfellow, Yoshua Bengio and Aaron Courville (2016)
\newblock Deep Learning
\newblock \emph{\url{http://www.deeplearningbook.org/}}
%%%%%%%%%%%%%%%%%%%%%%%%%%%%%%%%%%
\bibitem[Andrej Karpathy., 2015]{1} Andrej Karpathy (2015)
\newblock The Unreasonable Effectiveness of Recurrent Neural Networks
\newblock \emph{\url{http://karpathy.github.io/2015/05/21/rnn-effectiveness/}}
%%%%%%%%%%%%%%%%%%%%%%%%%%%%%%%%%%
\end{thebibliography}
}
\end{vbframe}
%%%%%%%%%%%%%%%%%%%%%%%%%%%%%%%%%%%%%%%%%%%%%%%%%%%%%%%%%%%%%%%%%%
%%%%%%%%%%%%%%%%%%%%%%%%%%%%%%%%%%%%%%%%%%%%%%%%%%%%%%%%%%%%%%%%%%

\endlecture
\end{document}