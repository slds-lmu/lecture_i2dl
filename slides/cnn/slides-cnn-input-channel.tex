\input{../../style/preamble}
\input{../../latex-math/basic-math}
\input{../../latex-math/basic-ml}
\input{../../latex-math/ml-nn}

\newcommand{\titlefigure}{figure/gray.png} %modify picture
\newcommand{\learninggoals}{
  \item Input Channel for Gray Images
  \item Input Channels for Colored Images
  \item Output Dimension Calculation
}

\title{Deep Learning}
\date{}



\begin{document}

\lecturechapter{CNN: Input Channel}
\lecture{I2DL}
%%%%%%%%%%%%%%%%%%%%%%%%%%%%%%%%%%%%%%%%%%%%%%%%%%%%%%%%%%%%%%%%%%

\frametitle{Input Channel}

\begin{vbframe}{Input Channel}
         \begin{figure}
    \centering
    \includegraphics[width=4cm]{figure/gray.png}
  \end{figure}
    \begin{itemize}
    
       \item An image consists of the smallest indivisible segments called pixels and every pixel has a strength often known as the pixel intensity. 
       
       \item A grayscale image has a single input channel and the value of each pixel represents the amount of light.
       
       \item A grayscale value can lie between 0 to 255, where 0 value corresponds to black and 255 to white.
       
       
    \end{itemize}

\end{vbframe}

\begin{vbframe}

 \begin{figure}
    \centering
    \includegraphics[width=7cm]{figure/RGB.jpeg}
  \end{figure}

 \begin{figure}
    \centering
    \includegraphics[width=5cm]{figure/RGB-1.png}
    \caption{\tiny Image source: Computer Vision Primer: How AI Sees An Imag eKishan Maladkar's Blog)}
  \end{figure}

\begin{itemize}
       \item A colored digital image usually comes with three color channels, i.e. the Red-Green-Blue channels or RGB values.   
        \item Each pixel can be represented by a vector of three numbers (each ranging from 0 to 255) for the three primary color channels.
      
\end{itemize}
 \begin{figure}
    \centering
    \includegraphics[width=5cm]{figure/1channel.png}
    \caption{\tiny A CNN taking a grayscale image as input.}
  \end{figure}


 \begin{figure}
    \centering
    \includegraphics[width=5cm]{figure/3channel.png}
    \caption{\tiny A CNN processing a colored images where each of the color spectrums serve as input. (Source: Chaitanya Belwal's Blog)}
  \end{figure}
  
  
\end{vbframe}
\begin{vbframe}

 \begin{figure}
    \centering
    \includegraphics[width=5cm]{figure/3channel.png}
  \end{figure}

In this CNN:
    \begin{itemize}
       \item there are 3 input channels, represented as 4x4 input matrices, 
       \item one 2x2 filter (also known as kernel), 
       \item a single ReLu layer,
       \item a single pooling layer (which applies the MaxPool function),
       \item and a single fully connected (FC) layer.
    \end{itemize}

    \begin{itemize}
       \item The elements of the filter matrix are equivalent to the unit weights in a standard NN and will be updated during the backpropagation phase.
       \item Assuming a stride of 2 with no padding, the output size of the convolution layer is determined by the following equation:
       \item $ O = \frac{I - K + 2.P}{S} + 1$ where: 
    \begin{itemize}
       \item O: is the dimension (rows and columns) of the output square matrix, 
       \item I: is the dimension (rows and columns) of the input square matrix,
       \item K: is the dimension (rows and columns) of the filter (kernel) square matrix, 
       \item P: is the number of pixels (cells) of padding added to each side of the input,
       \item S: is the stride, or the number of cells skipped each time the kernel is slided.
    \end{itemize}
    \end{itemize}

 \begin{figure}
    \centering
    \includegraphics[width=5cm]{figure/3channel.png}
  \end{figure}

Inserting the values shown in the figure into the equation yields

\begin{align} 
O= \frac{I - K + 2.P}{S} + 1&={(4 - 2 + 2.0)\over 2} + 1\\ 
&=2. 
\end{align}

\end{vbframe}



%%%%%%%%%%%%%%%%%%%%%%%%%%%%%%%%%%%%%%%%%%%%%%%%%%%%%%%%%%%%%%%%%%
\endlecture
\end{document}